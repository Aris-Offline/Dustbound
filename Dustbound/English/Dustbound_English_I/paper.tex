\documentclass{exam-zh}

\usepackage{xeCJK}
\usepackage{background}
\usepackage{needspace}
\usepackage{listing}

\examsetup{
  page = {
    foot-content = 「赴尘杯」英语试题第 ; 页(共 ; 页)
  }
}
\backgroundsetup{
  scale=20,
  color=black!10,
  opacity=1,
  angle=45,
  position=current page.center,
  contents={赴尘杯}
}

\title{「赴尘杯」2025 年普通高中学业水平选择性考试适应性演练}
\subject[3.5em]{英语 I}

\begin{document}

\information{
  姓名\underline{\hspace{6em}},
  准考证号\underline{\hspace{15em}},
  座位号\underline{\hspace{6em}}
}

\vspace{1\baselineskip}

\secret

\maketitle

本试卷共 16 页,67 题。全卷满分 150 分。考试用时 120 分钟。

\begin{notice}[][itemsep=0pt, parsep=0pt, topsep=0pt, partopsep=0pt]
\item 答卷前,考生务必将自己的姓名、考生号等填写在答题卡上。
\item 回答选择题时,选出每小题答案后,用铅笔把答题卡上对应题目的答案标号涂黑,如需改动,用橡皮擦干净后,再选涂其他答案标号。回答非选择题时,将答案写在答题卡上,写在本试卷上无效。
\item 考试结束后,将本试卷和答题卡一并交回。
\end{notice}

\warning{持有者应知晓本试题仍处于早期构建中,其排版、内容或分发形式可能会在构建过程中发生较大的变动,恕不另行通知。}

\begin{flushleft}
  {\bfseries 第二部分 \hspace{0.5em} 阅读理解(共两节,满分 50 分)}

{\bfseries 第一节(共 15 小题;每小题 2.5 分,满分 37.5 分)}
\end{flushleft}

阅读下列短文,从每题所给的 A、B、C 和 D 四个选项中,选出最佳答案。

\needspace{4\baselineskip}
{\centering \bfseries A \par}

{\centering \bfseries Unified Extensible Firmware Interface / Secure Boot \par}

Secure Boot is a security feature found in the UEFI standard, designed to add a layer of protection to the pre-boot process: by maintaining a cryptographically (使用密码学方法地) signed list of binaries authorized or forbidden to run at boot, it helps in improving the confidence that the machine core boot components (boot manager, kernel, initramfs) have not been tampered with.

As such it can be seen as a continuation or complement to the efforts in securing one's computing environment, reducing the attack surface that other software security solutions such as system encryption cannot easily cover, while being totally distinct and not dependent on them. Secure Boot just stands on its own as a component of current security practices, with its own set of pros and cons.

\needspace{2\baselineskip}
\textbf{I. Before Booting the OS}

(This content is missing and needs to be completed in Question 3.)

\needspace{2\baselineskip}
\textbf{II. Booting an installation medium}

In order to boot an installation medium in a Secure Boot system, you will need to either disable Secure Boot or modify the image in order to add a signed boot loader.

\begingroup
\leftskip=2em

\needspace{2\baselineskip}
\textbf{i. Disabling Secure Boot}

The Secure Boot feature can be disabled via the UEFI firmware interface. How to access the firmware configuration is described in \underline{Before booting the OS}. Some motherboards (this is the case in a Packard Bell laptop and recent Xiaomi laptops) only allow to disable secure boot if you have set an administrator password that can be removed afterwards. See also \underline{Rod Smith's Disabling Secure Boot}.

\needspace{2\baselineskip}
\textbf{ii. Editing the installation medium}

If you are using a USB flash installation medium, then it is possible to manually edit the EFI system partition on the medium to add support for Secure Boot.

\par
\endgroup

\needspace{2\baselineskip}
\textbf{III. Implementing Secure Boot}

Using a signed boot loader means using a boot loader signed with Microsoft's key. There are two known signed boot loaders: PreLoader and shim. Their purpose is to chainload other EFI binaries (usually boot loaders). Since Microsoft would never sign a boot loader that automatically launches any unsigned binary, PreLoader and shim use an allowlist called Machine Owner Key list, abbreviated MokList. If the SHA256 hash of the binary (Preloader and shim) or key the binary is signed with (shim) is in the MokList they execute it, if not they launch a key management utility which allows enrolling the hash or key.

\begingroup
\leftskip=2em

\needspace{2\baselineskip}
\textbf{i. PreLoader}

When run, PreLoader tries to launch \texttt{loader.efi}. If the hash of \texttt{loader.efi} is not in MokList, PreLoader will launch \texttt{HashTool.efi}. In HashTool you must enroll the hash of the EFI binaries you want to launch, that means your boot loader (\texttt{loader.efi}) and kernel. Each time you update any of the binaries (e.g. boot loader or kernel) you will need to enroll their new hash.

\needspace{2\baselineskip}
\textbf{(1) Set up PreLoader}

\textbf{NOTE:} \texttt{PreLoader.efi} and \texttt{HashTool.efi} in \texttt{efitools} package are not signed, so their usefulness is limited. You can get a signed \texttt{PreLoader.efi} and \texttt{HashTool.efi} from \texttt{preloader-signed}\textsuperscript{AUR} or download them manually.

Install \texttt{preloader-signed}\textsuperscript{AUR} and copy \texttt{PreLoader.efi} and \texttt{HashTool.efi} to the boot loader directory, then copy over the boot loader binary and rename it to \texttt{loader.efi}. Finally, create a new NVRAM entry to boot \texttt{PreLoader.efi}. After these steps, this entry should be added to the list as the first to boot; check with the \texttt{efibootmgr} command and adjust the boot-order if necessary.

\needspace{2\baselineskip}
\textbf{(2) Fallback}

If there are problems booting the custom NVRAM entry, copy \texttt{HashTool.efi} and \allowbreak \texttt{loader.efi} to the default loader location booted automatically by UEFI systems. Copy over \texttt{PreLoader.efi} and rename it.

For particularly intransigent UEFI implementations, copy \texttt{PreLoader.efi} to the default loader location used by Windows systems. Then as before, copy \texttt{HashTool.efi} and \texttt{loader.efi} to \texttt{esp/EFI/Microsoft/Boot/}.

If dual-booting with Windows, backup the original \texttt{bootmgfw.efi} first as replacing it may cause problems with Windows updates. When the system starts with Secure Boot enabled, follow the steps above to enroll \texttt{loader.efi} and \texttt{/vmlinuz-linux} (or whichever kernel image is being used).

\needspace{2\baselineskip}
\textbf{(3) How to use while booting?}

A message will show up that says \textit{Failed to Start loader... I will now execute HashTool.} To use HashTool for enrolling the hash of \texttt{loader.efi} and \texttt{vmlinuz.efi}, follow these steps. These steps assume titles for a remastered archiso installation media. The exact titles you will get depend on your boot loader setup.

\begin{itemize}
\item Select \textit{OK}.

\item In the HashTool main menu, select \textit{Enroll Hash}, choose \texttt{\\loader.efi} and confirm with \textit{Yes}. Again, select \textit{Enroll Hash} and \texttt{archiso} to enter the archiso directory, then select \texttt{vmlinuz.efi} and confirm with \textit{Yes}. Then choose \textit{Exit} to return to the boot device selection menu.

\item In the boot device selection menu choose \textit{Arch Linux archiso x86\_64 UEFI CD}.
\end{itemize}

\needspace{2\baselineskip}
\textbf{ii. shim}

When run, shim tries to launch \texttt{grubx64.efi}. If MokList does not contain the hash of \texttt{grubx64.efi} or the key it is signed with, shim will launch MokManager (\texttt{mmx64.efi}). In MokManager you must enroll the hash of the EFI binaries you want to launch (your boot loader (\texttt{grubx64.efi}) and kernel) or enroll the key they are signed with.

Using hash is simpler, but each time you update your boot loader or kernel you will need to add their hashes in MokManager. With MOK you only need to add the key once, but you will have to sign the boot loader and kernel each time it updates.

\par
\endgroup

\needspace{2\baselineskip}
\textbf{IV. Protecting Secure Boot}

The only way to prevent anyone with physical access from disabling Secure Boot is to protect the firmware settings with a password. Most UEFI firmwares provide such a feature, usually listed under the "Security" section in the firmware settings.

\needspace{4\baselineskip}
\begin{question}
Which of the following statements about Secure Boot is INCORRECT?
\begin{choices}
\item Secure Boot represents a critical security mechanism embedded within the UEFI standard.

\item Secure Boot functions by enforcing a policy that restricts the execution of boot-time binaries to those that are cryptographically signed and thereby verified as authentic.

\item Secure Boot ensures the integrity of core boot components and the efficacy of system encryption relies on the presence or state of Secure Boot.

\item Secure Boot operates independently of other security measures and provides an additional layer of defense against pre-boot attacks.
\end{choices}
\end{question}

\needspace{4\baselineskip}
\begin{question}
Which of the following statements about the possible usage and configuration of Secure Boot is CORRECT?

\begin{choices}
\item Recent iterations of Xiaomi laptops give users access to the configuration of Secure Boot within the firmware only if the administrator password remains active after setting up. Users can possibly set the password in the "Security" section in the firmware settings.

\item Developers of Linux distributions have the capability to assist users to employ PreLoader which includes HashTool, a mechanism which facilitates the integration of the keys utilized for digitally signing EFI binary files into the Machine Owner Key List (MokList) of the user's laptop.

\item When configuring Secure Boot using PreLoader and encountering issues with booting a custom NVRAM entry, it is always necessary to copy \texttt{HashTool.efi} and \texttt{loader.efi} to the UEFI system's default automatic boot loader location. Subsequently, \texttt{PreLoader.efi} must be copied to the default boot loader location used by Windows systems.

\item During the boot process, suppose HashTool is used to enroll the hashes of \texttt{loader.efi} and \texttt{vmlinuz.efi}. Start by selecting \textit{OK} to begin. Enroll the hash for \texttt{loader.efi} by choosing \textit{Enroll Hash} and confirming with \textit{Yes}, and then repeat the process for \texttt{vmlinuz.efi}. Finally, exit HashTool to return to the boot device selection menu, and select the appropriate Arch Linux boot option (such as \textit{Arch Linux archiso x86\_64 UEFI CD}), depending on your boot loader configuration, to start the system.
\end{choices}
\end{question}

\needspace{4\baselineskip}
\begin{question}
Which content is possibly included in \underline{Before booting the OS}?

\needspace{3\baselineskip}
A.
\begin{quote}
  At this point, one has to look at the firmware setup. If the machine was booted and is running, in most cases it will have to be rebooted.

  You may access the firmware configuration by pressing a special key during the boot process. The key to use depends on the firmware. It is usually one of Esc, F2, Del or possibly another Fn key. Sometimes the right key is displayed for a short while at the beginning of the boot process. The motherboard manual usually records it. You might want to press the key, and keep pressing it, immediately following powering on the machine, even before the screen actually displays anything.

  After entering the firmware setup, be careful not to change any settings without prior intention. Usually there are navigation instructions, and short help for the settings, at the bottom of each setup screen. The setup itself might be composed of several pages. You will have to navigate to the correct place. The interesting setting might be simply denoted by secure boot, which can be set on or off.
\end{quote}

\needspace{3\baselineskip}
B.
\begin{quote}
  There are certain conditions making for an ideal setup of Secure boot:

  \begin{itemize}
    \item UEFI considered mostly trusted (despite having some well known criticisms and vulnerabilities) and necessarily protected by a strong password.
    \item Default manufacturer/third party keys are not in use, as they have been shown to weaken the security model of Secure Boot by a great margin.
    \item UEFI directly loads a user-signed EFI boot stub-compatible unified kernel image (no boot manager), including microcode (if applicable) and initramfs so as to maintain throughout the boot process the chain of trust established by Secure Boot and reduce the attack surface.
    \item Use of full drive encryption, so that the tools and files involved in the kernel image creation and signing process cannot be accessed and tampered with by someone having physical access to the machine.
    \item Some further improvements may be obtained by using a TPM.
  \end{itemize}

  A simple and fully self-reliant setup is described in \underline{Using your own keys}, while \allowbreak \underline{Using a signed boot loader} makes use of intermediate tools signed by a third-party.
\end{quote}

\needspace{3\baselineskip}
C.
\begin{quote}
  Replacing the platform keys with your own can end up bricking hardware on some machines, including laptops, making it impossible to get into the firmware settings to rectify the situation. This is due to the fact that some device (e.g GPU) firmware (OpROMs), that get executed during boot, are signed using Microsoft 3rd Party UEFI CA certificate or vendor certificates. This is the case in many Lenovo Thinkpad X, P and T series laptops which uses the Lenovo CA certificate to sign UEFI applications and firmware.

  \textbf{DO NOT} copy the \texttt{noPK.auth} file to the EFI system partition (ESP) of your PC! If you do this, and someone e.g. steals your PC, this person can delete the personal Platform Key in the UEFI Secure Boot firmware again, turn on "Setup Mode" on your PC again and replace your Secure Boot Keys (PK, KEK, db, dbx) with their own Platform Key, thereby defeating the whole purpose of UEFI Secure Boot. Only you should be able to replace the Platform Key, so only you should have access to the \texttt{noPK.auth} file. Therefore keep the \texttt{noPK.auth} file in a secret, safe place where only you have access to.
\end{quote}

\needspace{3\baselineskip}
D.
\begin{quote}
  Point the current boot device to the one which has the Arch Linux installation medium. Typically it is achieved by pressing a key during the POST phase, as indicated on the splash screen. Refer to your motherboard's manual for details.

  When the installation medium's boot loader menu appears,

  \begin{itemize}
    \item if you used the ISO, select Arch Linux install medium and press Enter to enter the installation environment.

    \item if you used the Netboot image, choose a geographically close mirror from Mirror menu, then select Boot Arch Linux and press Enter.
  \end{itemize}

  You will be logged in on the first virtual console as the root user, and presented with a Zsh shell prompt.

  To switch to a different console—for example, to view this guide with Lynx alongside the installation—use the Alt+arrow shortcut. To edit configuration files, mcedit, nano and vim are available. See \texttt{pkglist.x86\_64.txt} for a list of the packages included in the installation medium.
\end{quote}
\end{question}

\needspace{4\baselineskip}
{\centering \bfseries B \par}

{\centering \bfseries Passerine / Dream SMP Fanwork - Chapter 1 \par}

“You’re a bastard,” Techno said playfully, even as the voices screamed, run, run, run. He took Philza’s offered hand and pulled himself up beside the man that he was sure could have cut him in two, no matter how dulled the sword’s edge was. As Philza patiently moved Techno through all the things he’d done wrong (small things like foot placement and his hilt grip being an inch off), Techno found it equal parts amusing and frightening that despite his eons of bloody fighting, it took only a few minutes of sparring for Philza to find flaws in his technique. But then again, Techno’s technique wasn’t particularly polished; it took only one brutal swing to fell most people. Something told him that Philza would be harder to kill than that.

They conquered nations, he and his golden-haired friend. They were bathed in glory, twin gods shining in the middle of a bloody fields. But as their empire grew, so did their enemies. They came in droves, day after day, and before long Techno had forgotten what peace tasted like. The days were long and the nights were longer; every flicker of movement was a spy in the shadows, every ally was a potential traitor, every word was a declaration of war. Their home had become a target for a thousand armies.

Through it all, his one constant was Philza—until he wasn’t. Technoblade simply looked up one day from a map detailing enemy lines and realized he’d been talking to empty air. He had no idea how long he’d been alone, sitting in a dusty library with stale tea untouched in the corner. He had no idea if Philza ever said he was leaving, or if he simply went as he arrived suddenly, swiftly, like a snowstorm.

Afterwards, there was hardly any point in maintaining the empire. The voices were getting bored, anyway. They wanted more. A fresh fight, more stories. So Techno took his sword and his shield, and abandoned ship. He’d done it a million times before, but the thought of a chess board lying unused in a crumbling castle made him feeling something close to regret.

Technoblade wandered the world, trying to appease the voices. Neither of them were ever satisfied. No matter how much chaos he dealt, there was always more work to be done. So he worked. He had no idea for how long. All he remembered from that bloody time was a sense of unfulfillment, like a story had been left unfinished halfway through. Years. Decades. Maybe more. It hardly mattered.

In the end, he knew, it would all be the same. The world would end, and he would remain—always fighting, always alone. He’d seen it in his dreams, in the visions that plagued him. The world would end, and he would remain.

\needspace{4\baselineskip}
\begin{question}
What is the relationship between Technoblade and Philza in the story?
\begin{choices}
  \item They are twin gods who have conquered nations and are bathed in glory.
  \item They are friends who have fought side by side and have grown apart over time.
  \item They are enemies who have fought each other for eons and have never been able to kill each other.
  \item They are allies who have been betrayed by their enemies and have been forced to abandon their empire.
\end{choices}
\end{question}

\needspace{4\baselineskip}
\begin{question}
What is the main reason for Technoblade to abandon his empire and wander the world?
\begin{choices}
  \item He was bored with the empire and wanted to find new challenges.
  \item He was betrayed by his allies and was forced to leave the empire.
  \item He was haunted by the voices in his head and was unable to find peace.
  \item He was tired of fighting and wanted to find a way to end the world.
\end{choices}
\end{question}

\needspace{4\baselineskip}
\begin{question}
What is the author's attitude towards the relationship between Technoblade and Philza?
\begin{choices}
  \item Neutral.
  \item Sympathetic.
  \item Critical.
  \item Humorous.
\end{choices}
\end{question}

\needspace{4\baselineskip}
\begin{question}
What is the main theme of the story?
\begin{choices}
  \item The importance of friendship and loyalty.
  \item The inevitability of conflict and betrayal.
  \item The futility of power and glory.
  \item The loneliness of eternal existence.
\end{choices}
\end{question}



\end{document}

