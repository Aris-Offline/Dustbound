\documentclass{exam-zh}

\usepackage{xeCJK}
\usepackage{background}
\usepackage{needspace}
\usepackage{listing}

\examsetup{
  page = {
    foot-content = 「赴尘杯」思想政治试题第 ; 页(共 ; 页)
  }
}
\backgroundsetup{
  scale=20,
  color=black!10,
  opacity=1,
  angle=45,
  position=current page.center,
  contents={赴尘杯}
}

\newenvironment{kaiti-indented}{
  \parindent=2em % 设置首行缩进为两个汉字宽度
  \CJKfamily{zhkai} % 切换到楷体
}{
}

\title{2025 年陕西省高中学业水平选择性考试}
\subject[3.5em]{思想政治}

\begin{document}

\secret

\maketitle

本试卷共 16 页,67 题。全卷满分 150 分。考试用时 120 分钟。

\begin{notice}[][itemsep=0pt, parsep=0pt, topsep=0pt, partopsep=0pt]
\item 答卷前,考生务必将自己的姓名、准考证号填写在答题卡上。
\item 作答选择题时,选出每小题答案后,用 2B 铅笔在答题卡上对应题目选项的答案信息点涂黑,如需改动,用橡皮擦干净后,再选涂其他答案;作答非选择题时,用 0.5mm 黑色墨水签字笔在答题卡的对应题目位置作答,写在试题卷、草稿纸和答题卡上的非答题区域均无效。
\item 考试结束后,将本试卷和答题卡一并上交。
\end{notice}

\begin{flushleft}
{\bfseries 一、选择题:本题共 16 小题,每小题 3 分,共 48 分。在每小题给出的四个选项中,只有一个选项是正确的。请将正确的选项填涂在答题卡相应的位置上。)}
\end{flushleft}
\vspace{-1em}

\needspace{3\baselineskip}
\begin{question}
习近平总书记在 2025 年新年贺词中指出:党的二十届三中全会胜利召开,吹响进一步全面深化改革的号角。我们乘着改革开放的时代大潮阔步前行,中国式现代化必将在改革开放中开辟更加广阔的前景。由此可见

\circlednumber{1} 改革开放是决定当代中国命运的关键一招

\circlednumber{2} 党的二十届三中全会标志着中国特色社会主义进入了新阶段

\circlednumber{3} 全面深化改革有利于推进中国式现代化发展

\circlednumber{4} 中国共产党坚持始终走在时代前列

\begin{choices}
\item \circlednumber{1}\circlednumber{2}
\item \circlednumber{1}\circlednumber{3}
\item \circlednumber{2}\circlednumber{3}
\item \circlednumber{3}\circlednumber{4}
\end{choices}
\end{question}

\needspace{3\baselineskip}
\begin{question}
第二次世界大战后,资本主义有三大新变化:即新自由主义化、金融深化、全球经济虚拟化,这三大变化导致资本主义的不平等问题日益严重、经济出现停滞的趋势、泡沫经济和金融危机频发。由此可见

\circlednumber{1} 资本主义政府难以实现经济宏观调控

\circlednumber{2} 资本主义社会的基本矛盾是生产资料私有制和生产社会化的矛盾

\circlednumber{3} 资本主义的不平等问题可能随着生产力的不断发展而得到解决

\circlednumber{4} 周期性的资本主义经济危机是资本主义社会无法克服的障碍

\begin{choices}
\item \circlednumber{1}\circlednumber{2}
\item \circlednumber{1}\circlednumber{4}
\item \circlednumber{2}\circlednumber{3}
\item \circlednumber{2}\circlednumber{4}
\end{choices}
\end{question}

\needspace{3\baselineskip}
\begin{question}
某校思想政治学习小组收集了我国不同时期领导人的经济论述。

\begin{kaiti-indented}
江泽民说:实践的发展和认识的深化,要求我们明确提出,我国经济体制改革的目标是建立社会主义市场经济体制,以利于进一步解放和发展生产力。

胡锦涛说:提高开放水平,着力构建充满活力、富有效率、更加开放、有利于科学发展的体制机制,为发展中国特色社会主义提供强大动力和体制保障。

习近平说:坚持和完善我国社会主义基本经济制度和分配制度,毫不动摇巩固和发展公有制经济,毫不动摇鼓励、支持、引导非公有制经济发展,使市场在资源配置中起决定性作用,更好发挥政府作用。
\end{kaiti-indented}

对以上论述的分析合理的是

\circlednumber{1} 社会主义市场经济体制的建立和完善体现了党与时俱进的思想路线

\circlednumber{2} 社会主义市场经济体制要求毫不动摇巩固公有制经济主体地位

\circlednumber{3} 认识具有无限性,追求真理永无止境

\circlednumber{4} 改革是社会主义社会实现社会变革的决定力量

\begin{choices}
\item \circlednumber{1}\circlednumber{3}
\item \circlednumber{2}\circlednumber{3}
\item \circlednumber{2}\circlednumber{4}
\item \circlednumber{3}\circlednumber{4}
\end{choices}
\end{question}

\needspace{3\baselineskip}
\begin{question}
从党的十四届三中全会提出要建立社会保障制度,到党的十六大明确地把“社会保障体系比较健全”作为全面建设小康社会的目标之一,再到党的十八大把社会保障全民覆盖作为全面建成小康社会的重要目标,我国在推进社会保障制度建设的道路上行稳致远。以下说法合理的是

\circlednumber{1} 社会保障是政府承担主要责任的“安全网”

\circlednumber{2} 社会保障的运行方式是风险分摊和责任共担

\circlednumber{3} 只有参与劳动的社会成员才有权享受社会保险提供的保障

\circlednumber{4} 我国要最终建设覆盖全体人民的社会保障体系

\begin{choices}
\item \circlednumber{1}\circlednumber{2}
\item \circlednumber{1}\circlednumber{3}
\item \circlednumber{2}\circlednumber{3}
\item \circlednumber{2}\circlednumber{4}
\end{choices}
\end{question}

\needspace{3\baselineskip}
\begin{question}
自 2025 年 1 月 1 日起各大银行将执行新的存量房贷利率,且绝大多数借款人不需要提交额外申请或办理相关手续。以贷款额 100 万元、等额本息、30 年期限为例,在房贷利率从 3.9\% 下调至 3.3\% 后,每月利息支出可节省 300 多元。由此可见

\circlednumber{1} 中央人民政府通过调节财政政策刺激居民消费增长

\circlednumber{2} 房贷利率随市场波动调节体现了市场在资源配置中的决定性作用

\circlednumber{3} 中国人民银行积极促进金融业协调健康发展

\circlednumber{4} 我国政府坚持把人民幸福作为发展的最终目的

\begin{choices}
\item \circlednumber{1}\circlednumber{2}
\item \circlednumber{1}\circlednumber{3}
\item \circlednumber{2}\circlednumber{3}
\item \circlednumber{2}\circlednumber{4}
\end{choices}
\end{question}

\end{document}
